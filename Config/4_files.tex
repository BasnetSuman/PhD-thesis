%!TEX root = ../template.tex
%%%%%%%%%%%%%%%%%%%%%%%%%%%%%%%%%%%%%%%%%%%%%%%%%%%%%%%%%%%%%%%%%%%%
%% Config/4_files.tex
%% NOVA thesis configuration file
%%%%%%%%%%%%%%%%%%%%%%%%%%%%%%%%%%%%%%%%%%%%%%%%%%%%%%%%%%%%%%%%%%%%

\typeout{NT FILE Config/4_files.tex}%

%=============================================================================
% THE FILES
%=============================================================================

%%------------------------------------------------------------
%% Except for the bibliography, all the other FILE NAMES below
%%     inside braces must correspond to a file with extension
%%     ".tex" and located in the "Chapters" folder
%%------------------------------------------------------------

%%------------------------------------------------------------
% BibTeX bibiography files and customization.
%   May be used multiple times with a single file nae each time.
%   syntax: \ntaddfile{bib}{filename.bib}
\ntaddfile{bib}{bibliography.bib}
%\ntaddfile{bib}{anotherbibfile.bib}

%%------------------------------------------------------------
% File with the dedicatory text. Will only be considered for final documents,
%   i.e., "bsc", "msc" and "phd", otherwise, it will be silently ignored
%   syntax: \ntaddfile{dedicatory}{filename}
\ntaddfile{dedicatory}{dedicatory}

%%------------------------------------------------------------
% File with the acknowledgments text. Will only be considered for final documents,
%   i.e., "bsc", "msc" and "phd", otherwise, it will be silently ignored
%   syntax: \ntaddfile{acknowledgements}{filename}
\ntaddfile{acknowledgements}{acknowledgements}

%%------------------------------------------------------------
% File with the quote text. Will only be considered for final documents,
%   i.e., "bsc", "msc" and "phd", otherwise, it will be silently ignored
%   syntax: \ntaddfile{quote}{filename}
\ntaddfile{quote}{quote}

%%------------------------------------------------------------
% Abstract files in multiple languages
%   syntax: \ntaddfile{abstract}[language]{filename}
\ntaddfile{abstract}[pt]{abstract-pt}  % Abstract in Portuguese
\ntaddfile{abstract}[en]{abstract-en}  % Abstract in English
% \ntaddfile{abstract}[fr]{abstract-fr}  % Abstract in French
% \ntaddfile{abstract}[gr]{abstract-gr}  % Abstract in Greek
% \ntaddfile{abstract}[pt-ext]{abstract-pt-ext}  % Extended abstract in Portuguese (e.g., for ulisboa/fmv)

%%------------------------------------------------------------
% Lists of Glossary, Acronyms and Symbols
%   syntax: \ntaddfile{glossaries}[list-type]{filename}
\ntaddfile{glossaries}[glossary]{glossary}
\ntaddfile{glossaries}[acronym]{acronyms}
\ntaddfile{glossaries}[symbols]{symbols}
% \ntaddfile{glossaries}[symbols]{chemical}

%%------------------------------------------------------------
% Main text - document chapters
%   syntax: \ntaddfile{chapter}[doc-type-list]{filename}
%         If “doc-type-list” is omitted, add chapter to all doc types
%   examples:  \ntaddfile{chapter}{my-chapter} % for all documents
%              \ntaddfile{chapter}[phdplan]{my-chapter} % just for phd documents
%              \ntaddfile{chapter}[phd,phdplan]{my-chapter} % just both phd and phdplan documents
\ntaddfile{chapter}{chapter1}
\ntaddfile{chapter}{chapter2}
\ntaddfile{chapter}{chapter3}
% \ntaddfile{chapter}{chapter4}
% \ntaddfile{chapter}{chapter-last}

%%------------------------------------------------------------
% Text appendices
% syntax: \ntaddfile{appendix}{filename}
\ntaddfile{appendix}{appendix1}
\ntaddfile{appendix}{appendix2}

%%------------------------------------------------------------
% Text annexes
% syntax: \ntaddfile{annex}{filename}
\ntaddfile{annex}{annex1}
% \ntaddfile{annex}{annex2}
% \ntaddfile{annex}{annex2}

%%------------------------------------------------------------
% User defined cover instead of standard one
%   syntax: \ntaddfile{cover}[1,N,spine]{filename}
% \ntaddfile{cover}[1]{cover-front}  % front cover
% \ntaddfile{cover}[N]{cover-back}   % back cover
% \ntaddfile{cover}[spine]{filename} % book spine
%
% To disable printing the second cover, use
% \ntsetup{print/secondcover=false}%
