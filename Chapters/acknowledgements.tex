%!TEX root = ../template.tex
%%%%%%%%%%%%%%%%%%%%%%%%%%%%%%%%%%%%%%%%%%%%%%%%%%%%%%%%%%%%%%%%%%%%
%% acknowledgements.tex
%% NOVA thesis document file
%%
%% Text with acknowledgements
%%%%%%%%%%%%%%%%%%%%%%%%%%%%%%%%%%%%%%%%%%%%%%%%%%%%%%%%%%%%%%%%%%%%

\typeout{NT LOADING FILE acknowledgements.tex}

\begin{ntacknowledgements}

Acknowledgments are personal text and should be a free expression of the author.

However, without any intention of conditioning the form or content of this text, I would like to add that it usually starts with academic thanks (instructors, etc.); then institutional thanks (Research Center, Department, Faculty, University, FCT / MEC scholarships, etc.) and, finally, the personal ones (friends, family, etc.).

But I insist that there are no fixed rules for this text, and it must, above all, express what the author feels.

\end{ntacknowledgements}