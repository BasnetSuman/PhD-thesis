%!TEX root = ../template.tex
%%%%%%%%%%%%%%%%%%%%%%%%%%%%%%%%%%%%%%%%%%%%%%%%%%%%%%%%%%%%%%%%%%%%
%% abstrac-pt.tex
%% NOVA thesis document file
%%
%% Abstract in Portuguese
%%%%%%%%%%%%%%%%%%%%%%%%%%%%%%%%%%%%%%%%%%%%%%%%%%%%%%%%%%%%%%%%%%%%

\typeout{NT FILE abstrac-pt.tex}%

Independentemente da língua em que a dissertação está escrita, geralmente esta contém pelo menos dois resumos: um resumo na mesma língua do texto principal e outro resumo numa outra língua.

A ordem dos resumos varia de acordo com a escola. Se a sua escola tiver regulamentos específicos sobre a ordem dos resumos, o template (\LaTeX) \gls{novathesis} irá respeitá-los. Caso contrário, a regra padrão no template \gls{novathesis} é ter em primeiro lugar o resumo \emph{no mesmo idioma do texto principal} e depois o resumo \emph{no outro idioma}. Por exemplo, se a dissertação for escrita em português, a ordem dos resumos será primeiro o português e depois o inglês, seguido do texto principal em português. Se a dissertação for escrita em inglês, a ordem dos resumos será primeiro em inglês e depois em português, seguida do texto principal em inglês.
%
No entanto, esse pedido pode ser personalizado adicionando um dos seguintes ao arquivo \verb+5_packages.tex+.

\begin{verbatim}
    \abstractorder(<MAIN_LANG>):={<LANG_1>,...,<LANG_N>}
\end{verbatim}

Por exemplo, para um documento escrito em Alemão com resumos em Alemão, Inglês e Italiano (por esta ordem), pode usar-se:
\begin{verbatim}
    \ntsetup{abstractorder={de={de,en,it}}}
\end{verbatim}

Relativamente ao seu conteúdo, os resumos não devem ultrapassar uma página e frequentemente tentam responder às seguintes questões (é imprescindível a adaptação às práticas habituais da sua área científica):

\begin{enumerate}
  \item Qual é o problema?
  \item Porque é que é um problema interessante/desafiante?
  \item Qual é a proposta de abordagem/solução?
  \item Quais são as consequências/resultados da solução proposta?
\end{enumerate}

% E agora vamos fazer um teste com uma quebra de linha no hífen a ver se a \LaTeX\ duplica o hífen na linha seguinte se usarmos \verb+"-+… em vez de \verb+-+.
%
% zzzz zzz zzzz zzz zzzz zzz zzzz zzz zzzz zzz zzzz zzz zzzz zzz zzzz zzz zzzz comentar"-lhe zzz zzzz zzz zzzz
%
% Sim!  Funciona! :)

% Palavras-chave do resumo em Português
% \begin{keywords}
% Palavra-chave 1, Palavra-chave 2, Palavra-chave 3, Palavra-chave 4
% \end{keywords}
\keywords{
  Primeira palavra-chave \and
  Outra palavra-chave \and
  Mais uma palavra-chave \and
  A última palavra-chave
}
% to add an extra black line
