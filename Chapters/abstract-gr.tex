%!TEX root = ../template.tex
%%%%%%%%%%%%%%%%%%%%%%%%%%%%%%%%%%%%%%%%%%%%%%%%%%%%%%%%%%%%%%%%%%%%
%% abstract-gr.tex
%% NOVA thesis document file
%%
%% Abstract in English([^%]*)
%%%%%%%%%%%%%%%%%%%%%%%%%%%%%%%%%%%%%%%%%%%%%%%%%%%%%%%%%%%%%%%%%%%%

\typeout{NT FILE abstract-gr.tex}%

\textbf{Αυτή είναι μια μετάφραση “Google Translate” από την αγγλική έκδοση! Διορθώσεις και διορθώσεις είναι ευπρόσδεκτες!}

Ανεξάρτητα από τη γλώσσα στην οποία συντάσσεται η διατριβή, συνήθως υπάρχουν τουλάχιστον δύο περιλήψεις: μια περίληψη στην ίδια γλώσσα με το κύριο κείμενο και μια περίληψη σε κάποια άλλη γλώσσα.

Η σειρά των περιλήψεων διαφέρει ανάλογα με το σχολείο. Εάν το σχολείο σας έχει συγκεκριμένους κανονισμούς σχετικά με τη σειρά των περιλήψεων, το πρότυπο \gls{novathesis} (\LaTeX) θα τους σέβεται. Διαφορετικά, ο προεπιλεγμένος κανόνας στο πρότυπο \gls{novathesis} είναι να έχετε στην πρώτη θέση την περίληψη στην \emph{την ίδια γλώσσα με το κύριο κείμενο} και μετά την περίληψη στην \emph{την άλλη γλώσσα}. Για παράδειγμα, εάν η διατριβή είναι γραμμένη στα πορτογαλικά, η σειρά των περιλήψεων θα είναι πρώτα πορτογαλικά και μετά στα αγγλικά, ακολουθούμενη από το κύριο κείμενο στα πορτογαλικά. Εάν η διατριβή είναι γραμμένη στα αγγλικά, η σειρά των περιλήψεων θα είναι πρώτα αγγλικά και στη συνέχεια πορτογαλικά, ακολουθούμενη από το κύριο κείμενο στα αγγλικά.
%
Ωστόσο, αυτή η σειρά μπορεί να προσαρμοστεί προσθέτοντας ένα από τα παρακάτω στο αρχείο \verb+5_packages.tex+.

\begin{verbatim}
    \ntsetup{abstractorder={<LANG_1>,...,<LANG_N>}}
    \ntsetup{abstractorder={<MAIN_LANG>={<LANG_1>,...,<LANG_N>}}}
\end{verbatim}

Για παράδειγμα, για ένα κύριο έγγραφο γραμμένο στα γερμανικά με περιλήψεις γραμμένες στα γερμανικά, αγγλικά και ιταλικά (με αυτή τη σειρά) χρησιμοποιήστε:
\begin{verbatim}
    \ntsetup{abstractorder={de={de,en,it}}}
\end{verbatim}

Όσον αφορά το περιεχόμενό του, οι περιλήψεις δεν πρέπει να υπερβαίνουν τη μία σελίδα και ενδέχεται να απαντούν στις ακόλουθες ερωτήσεις (είναι απαραίτητο να προσαρμοστείτε στις συνήθεις πρακτικές του επιστημονικού σας χώρου):

\begin{enumerate}
  \item Ποιο είναι το πρόβλημα;
  \item Γιατί είναι ενδιαφέρον/προκλητικό αυτό το πρόβλημα;
  \item Ποια είναι η προτεινόμενη προσέγγιση/λύση/συμβολή;
  \item Τι αποτελέσματα (συνέπειες/συνέπειες) από τη λύση;
\end{enumerate}

% Palavras-chave do resumo em Grego
% \begin{keywords}
% Keyword 1, Keyword 2, Keyword 3, Keyword 4, Keyword 5, Keyword 6, Keyword 7, Keyword 8, Keyword 9
% \end{keywords}
\keywords{
  Μία λέξη-κλειδί \and
  Μια άλλη λέξη-κλειδί \and
  Μια άλλη λέξη-κλειδί \and
  Μία λέξη-κλειδί ακόμα \and
  Η τελευταία λέξη-κλειδί
}
