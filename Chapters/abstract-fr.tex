%!TEX root = ../template.tex
%%%%%%%%%%%%%%%%%%%%%%%%%%%%%%%%%%%%%%%%%%%%%%%%%%%%%%%%%%%%%%%%%%%%
%% abstrac-fr.tex
%% NOVA thesis document file
%%
%% Abstract in French
%%%%%%%%%%%%%%%%%%%%%%%%%%%%%%%%%%%%%%%%%%%%%%%%%%%%%%%%%%%%%%%%%%%%

\typeout{NT FILE abstrac-fr.tex}%

\textbf{Ceci est une traduction “Google Translate” de la version anglaise ! Les correctifs et les corrections sont les bienvenus !}

Quelle que soit la langue dans laquelle la thèse est rédigée, il y a généralement au moins deux résumés : un résumé dans la même langue que le texte principal et un autre résumé dans une autre langue.

L'ordre des résumés varie selon l'école. Si votre école a des règles spécifiques concernant l'ordre des résumés, le modèle \gls{novathesis} (\LaTeX) les respectera. Sinon, la règle par défaut dans le modèle \gls{novathesis} est d'avoir en premier lieu le résumé en \emph{la même langue que le texte principal}, puis le résumé en \emph{l'autre langue}. Par exemple, si la thèse est rédigée en portugais, l'ordre des résumés sera d'abord le portugais puis l'anglais, suivi du texte principal en portugais. Si la thèse est rédigée en anglais, l'ordre des résumés sera d'abord en anglais puis en portugais, suivi du texte principal en anglais.
%
Cependant, cet ordre peut être personnalisé en ajoutant l'un des éléments suivants au fichier \verb+5_packages.tex+.

\begin{verbatim}
    \ntsetup{abstractorder={<LANG_1>,...,<LANG_N>}}
    \ntsetup{abstractorder={<MAIN_LANG>={<LANG_1>,...,<LANG_N>}}}
\end{verbatim}

Par exemple, pour un document principal rédigé en allemand avec des résumés rédigés en allemand, anglais et italien (dans cet ordre), utilisez:
\begin{verbatim}
    \ntsetup{abstractorder={de={de,en,it}}}
\end{verbatim}

Concernant son contenu, les résumés ne doivent pas dépasser une page et peuvent répondre aux questions suivantes (il est indispensable de s'adapter aux pratiques habituelles de votre domaine scientifique):

\begin{itemize}
% What's the problem?
  \item Quel est le problème?
% Why is it interesting?
  \item Pourquoi est-il intéressant?
% What's the solution?
  \item Quelle est la solution?
% What follows from the solution?
  \item Quels sont les résultats (conséquences) de la solution?
\end{itemize}

% Palavras-chave do resumo em Português
\begin{keywords}
Mots-clé 1, Mots-clé 2, Mots-clé 3
\end{keywords}
% to add an extra black line
